\documentclass[   % 
  final,          % 
  a4paper         % 
]{article}

\usepackage[a4paper,margin=2cm]{geometry}
\setcounter{secnumdepth}{0}

\usepackage{tikz}
\usetikzlibrary{matrix,calc,shapes}
\usepackage{amssymb}
\usepackage[image]{tikzinput}
\usepackage{standalone}

\begin{document}
\section{Usage examples}

This chapter is still in heavy developement.

\subsection{Basic molecular structure optimisations}

The molecular structure is often also referred to as \emph{geometry},
and hence the optimisation of the molecular structure is often referred
to as \emph{geometry optimisation}.

Here is a short checklist, flowchart, or list how to get from an initial
(guessed) structure to a final structure.

\begin{enumerate}
\item
  Create a molecular structure file in cartesian coordinates
  (\texttt{*.xyz}) with your preferred molecular editor, command line
  tool, or even text editor.\footnote{% 
  %If you (intend to) use Gaussview, please see some more \href{notes_gv_en.md}{notes on Gaussview}.
    Gaussview is unfortunately unable to write Xmol files, for whatever
    reason, this is not supported. Instead it can write a standard Gaussian
    input file, which you could set up ready-for-submission-complete with
    Gaussview. A lot of people do that, it is the GUI to Gaussian after all,
    the downside of this approach is that you tend to forget what you are
    doing.
    
    Not everything in those input files is necessary, and some of it is a
    terrible choice. For example, it usually includes the absolute paths to
    checkpoint files. If you decide to change anything (and this is quite
    likely) and then rerun a calculation, it will very likely fail, or
    overwrite existing things. If you are using Gaussview locally, but
    submit a calculation to the cluster, it is also likely, that the
    specified location does not exist.
    
    There are easy and not quite as easy workarounds:
    
    \begin{itemize}
    \item
      Use a different program.
    \item
      Edit the files on the cluster by hand, remove everything superflous
      with a text editor.
    \item
      Convert them with, e.g.~Molden. (Unfortunately Open Babel cannot do
      that.)
    \item
      Use my prepare script. It will search the inputfile for lines of the
      pattern \texttt{XX  +000.0000  -000.0000  +000.0000}
      (two letters followed by three signed real numbers).
    \item
      Alternatively, you can use a different program. (Seriously, it is
      worth mentioning it twice.)
    \item
      You can use Gaussian's \href{http://gaussian.com/newzmat/}{\texttt{newzmat}}
      utility to convert it.
    \end{itemize}
  }

  It needs a bit training to click together a reasonable guess,
  Chemcraft has a reasonably large fragments database, Molden has a few
  standards, but is (technically) able to read in fragments, too. For
  simple molecules you can sometimes find structures online, maybe even
  crystal structures. Another easy way is to search for it on
  \href{http://www.chemspider.com/}{ChemSpider},
  \href{https://pubchem.ncbi.nlm.nih.gov/search/}{PubChem}, or draw it
  with Chemdraw (etc.) and generate a
  \href{https://en.wikipedia.org/wiki/Simplified_molecular-input_line-entry_system}{SMILES}
  code. You can use \href{http://openbabel.org}{Open Babel} to create a
  guessed structure. For example, the SMILES for ethanol is
  \texttt{CCO}, see
  \href{https://pubchem.ncbi.nlm.nih.gov/compound/ethanol\#section=Canonical-SMILES}{PubChem
  CID 702}, then the command for Open Babel is:\newline
  \texttt{obabel -:'CCO' -oxyz --gen3d -Oopt.start.xyz}

  The option \texttt{-:} takes a SMILES as an input, the \texttt{-o}
  selects the output format (here chosen as Xmol, Cartesian
  coordinates), \texttt{-\/-gen3d} tells the program to generate 3D
  coordinates, and \texttt{-O} selects the output file to write
  on; see also the \href{http://openbabel.org/wiki/Main_Page}{Open Babel
  Wiki}.

  This process sometimes even works for more complex molecules; unfortunately
  not for organometallic (and similar) compounds, because SMILES does not describe
  ionic interactions.\footnote{%
    Some may work okay-ish, e.g. sodium ethanolate
    (\texttt{CC{[}O-{]}.{[}Na+{]}}), others not. You can try building the
    structure for the cation of
    \href{https://pubchem.ncbi.nlm.nih.gov/compound/407049}{tris(ethylenediamine)cobalt(III)
    chloride}:
    \texttt{C(C{[}NH-{]}){[}NH-{]}.C(C{[}NH-{]}){[}NH-{]}.C(C{[}NH-{]}){[}NH-{]}.{[}Co{]}.{[}Cl-{]}.{[}Cl-{]}.{[}Cl-{]}}.}
  Assume we did this, and we will call it \texttt{bp86svp.start.xyz}.
\item
  Create an input file. 
  A \emph{very simple} optimisation would probably use
  BP86/def2-SVP with mostly default parameters, or similar,\footnote{% 
  Choosing an appropriate level of theory is not often trivial, one has
  to consider a couple of things, like accuracy, performance, and speed.
  I have previously written quite a long article about that:
  \href{https://chemistry.stackexchange.com/a/27418/4945}{DFT Functional
  Selection Criteria}.

  The example command uses the DF-BP86/def2-SVP level of theory, but
  with something extra. Remember:

  \lstinline`g16.prepare -R'#P BP86/def2SVP/W06' -r'OPT(MaxCycles=100)' -j'bp86svp.opt' bp86svp.start.xyz`

  The \lstinline`-R` switch to \lstinline`g16.prepare` sets the basic route section to
  \texttt{\#P\ BP86/def2SVP/W06}, the different parts mean the following:

  \begin{itemize}
  \item
    The \texttt{\#P} selects verbose printing
    (\href{http://gaussian.com/route/?tabid=1}{G16 manual}), other
    options are \texttt{\#N} (normal) and \texttt{\#T} (terse).
  \item
    The method is selected as \texttt{BP86}, which selects the exchange
    functional \texttt{B} and the correlation functional \texttt{P86}.
    There are plenty of \href{http://gaussian.com/dft/}{Functionals
    implemented}.
  \item
    For the example I have chosen the def2-SVP basis set; please note
    that the keyword is without the dash. There are again many available
    in \href{http://gaussian.com/basissets/}{Gaussian}, and additional
    ones can be defined; that is something for more advanced users (and
    a topic for another day).
  \item
    Additionally, this route section requests density fitting (sometimes
    also referred to as \emph{resolution-of-the-identity}, or short RI,
    approximation) with the auxiliar basis set \texttt{W06} (see the
    \href{http://gaussian.com/basissets/?tabid=2}{Manual} for more),
    which should speed up the calculation. This is also available via
    the \href{http://gaussian.com/densityfit/}{\texttt{DensityFit}/\texttt{DenFit}}
    keyword, and therefore commonly abbreviated with DF in the level of
    theory.
  \end{itemize}

  The \lstinline`-r` switch to the script adds more keywords to the route
  section. In this specific case we'll request an optimisation
  (\href{http://gaussian.com/opt/}{\texttt{OPT}}) with 100 cycles at the
  most.

  The \lstinline`-j` switch selects a jobname for the calculation, and it
  will further be used to derive filenames for it.

  The last argument is the molecular structure file, here in the Xmol
  format. There are some other formats recognised, but that is also
  something for another day.}
  so you can just run the prepare script:

  \lstinline`g16.prepare -R'#P BP86/def2SVP/W06' -r'OPT(MaxCycles=100)' \`\par
  \lstinline`  -j'bp86svp.opt' bp86svp.start.xyz`

  This will create the file \texttt{bp86svp.opt.com}.

\item
  Submit the input file to the queue. 
  The following command will handle this:

  \lstinline`g16.submit -p12 -m4000  bp86svp.opt.com`

  This will create a modified inputfile \texttt{bp86svp.opt.gjf}, with
  the correct settings for Gaussian according to what you requested in
  the submit routine. It will also create (assuming it is set to
  \texttt{slurm-gen}) a script for the queueing system,
  \texttt{bp86svp.opt.slurm.bash}, which will be executed remotely.

  What sensible parameters (processors, memory, walltime) are comes 
  with experience. Obviously don't ask for more than you have 
  (\lstinline`-p12`, \lstinline`-m4000` is reasonable for medium sized molecule optimisations).
\item
  Once the calculation is done (you'll probably get an email), you will
  have a few output files.

  \begin{itemize}
  \item
    \texttt{bp86svp.opt.log}: the main output
  \item
    \texttt{bp86svp.opt.chk}: the checkpoint file (useful for later)
  \item
    \texttt{bp86svp.opt.slurm.bash.e012345}: the
    error file from the queueing system\footnote{%
      The number, here used as example \texttt{012345}, is the JobID
      assigned by the queueing system.}
  \item
    \texttt{bp86svp.opt.slurm.bash.o012345}: the
    standard output file from the queueing system
  \end{itemize}

  Check the main output file for errors. Everything should be fine when
  you find \emph{Normal termination} at the end (\texttt{tail bp86svp.opt.log}).
  You can (and should) also search for \emph{stationary point found}:\newline
  \texttt{grep 'Stationary point found' -B7 -A3  bp86svp.opt.log}.

  Don't forget to open it with a molecular viewer to make sure the
  structure is actually sensible.
\item
  Create a frequency calculation \textbf{in the same} directory:

  \lstinline`g16.freqinput bp86svp.opt.gjf`

  This creates \texttt{bp86svp.opt.freq.com} which you can submit:

  \lstinline`g16.submit -p12 -m24000  bp86svp.opt.freq.com`

  Frequency calculations are memory demanding, they will slow down if
  they do not have enough, about 2 GB per core should be okay.
\item
  You'll get similar output files like above. Open the \texttt{*.log}
  file with a molecular viewer, and look at the frequencies (or modes).
  If there are no imaginary modes, you have found a \emph{local}
  minimum, if there is exactly one mode, you found a transition state.
  Repeat the above steps as necessary if you didn't meet your target.
\item
  Look for the energy values and put them in tables to calculate
  barriers and stuff. You can use the post-processing script for that,
  or look through the manual by yourself:

  \lstinline`g16.getfreq -V3  bp86svp.opt.freq.log`
\item
  Finalise the calculation with creating a formatted checkpoint file and
  a coordinate file. The Gaussian checkpoint files are binary files and
  they are machine dependent (and fairly large). 
  That usually doesn't pose any problem, but
  it is good practice to create a pure text file anyway. The formatted
  checkpointfile can be used as an input for further analysis by many
  programs, therefore it is good to have in any case. 
  It can also be reverted to a binary checkpoint file, which means you
  can actually omit the checkpoint file while archiving.
  
  Writing an coordinate file with the optimised structure helps cutting 
  down loading times; it also gives you an indicator that the calculation 
  is done.

  You can do these two steps together with the following command:

  \lstinline`g16.chk2xyz  bp86svp.opt.freq.chk`

  The above command produces the formatted checkpoint file
  \texttt{bp86svp.opt.freq.fchk} and the coordinate file
  \texttt{bp86svp.opt.freq.xyz}. Alternatively you can use the script
  with the \lstinline`-a` switch to do that for every \texttt{*.chk} file
  in the current directory.
\end{enumerate}

Once you have done all the steps you can move on to interpreting the
results, which is where often the real work lies.

The following chart presents a basic diagramme of the above description,
it shows how the tools may interact with each other,
or can be chained after one another, respectively.

\tikzinput[width=0.99\linewidth]{./graphics/tikz-optimisation-flowchart}

\subsection{Transition state search}

The following chart attempts to do the same for the steps after a successful 
transition state optimisation.

(This part of the documentation is still \emph{very} much a work in progress.)

\tikzinput[width=0.99\linewidth]{./graphics/tikz-transitionstate-flowchart}

\end{document}

