\documentclass{standalone}

\usepackage{tikz}
\usetikzlibrary{matrix,calc,shapes}
\usepackage{amssymb}

\begin{document}
%define styles for nodes (from http://latex-cookbook.net/articles/flowchart/)
\tikzset{
  treenode/.style = {shape=rectangle, rounded corners,
                     draw, anchor=center,
                     text width=5cm, align=center, font=\rmfamily\normalsize,
                     top color=white, bottom color=blue!20,
                     inner sep=1ex},
  root/.style     = {treenode, font=\Large,
                     bottom color=red!30},
  decision/.style = {treenode, diamond, inner sep=0pt, aspect=2},
  env/.style      = {treenode, font=\ttfamily},
  archive/.style  = {env, bottom color=orange!40},
  result/.style   = {treenode, bottom color=green!40},
  finish/.style   = {root, bottom color=green!40},
  dummy/.style    = {circle,draw}
}

%\tikzset{
%  treenode/.style = {shape=rectangle, rounded corners,
%                     draw, anchor=center,
%                     text width=5cm, align=center, font=\rmfamily\normalsize,
%                     top color=white, bottom color=blue!20,
%                     inner sep=1ex},
%  root/.style     = {treenode, font=\Large,
%                     bottom color=red!30},
%  decision/.style = {treenode, diamond, inner sep=0pt, aspect=2},
%  env/.style      = {treenode, font=\ttfamily},
%  archive/.style  = {env, bottom color=orange!40},
%  result/.style   = {treenode, bottom color=green!40},
%  finish/.style   = {root, bottom color=green!40},
%  dummy/.style    = {circle,draw}
%}


\begin{tikzpicture}[-latex]
  \matrix (chart)
    [
      matrix of nodes,
      column sep      = 1cm,
      row sep         = 1cm,
      column 1/.style = {nodes={env}},
      column 2/.style = {nodes={archive}},
      column 3/.style = {nodes={result}}
    ]
    {
      %1
      |[root]| Start                           & \\
      %2
      <mol>.start.xyz                          & \\
      %3
      <method>.com                             & \\
      %4
      <method>.gjf, <method>.<queue>.bash      & %
        <method>.log, <method>.chk, (queue error/out) \\
      %5  
      |[decision]| Stationary point found?     & %
        <method>.xyz, <method>.fchk & %
        \(E_\mathrm{el}\)           \\
      %6
      <method>.freq.com                        & \\
      %7
      <method>.freq.gjf, <method>.freq.<queue>.bash & % 
        <method>.freq.log, <method>.freq.chk, (queue error/out) \\
      %8
                                               & %
        <method>.freq.xyz, <method>.freq.fchk  & 
        \(E_\mathrm{ZPE}, E_\mathrm{o}, \dots, G\)\\
      %9
      |[decision]| NImag \linebreak \( i \in \{0,1,...,\mathbb{N}\} \) & %
        |[result]| local minimum               & %
        |[decision]| target met?               \\
      %10
                                               & % 
        |[result]| transition state            & %
        |[decision]| target met?               \\
        %11
        &&&|[finish]| Done!\\
    };
  \draw
  (chart-1-1) edge node [right] {Chemcraft, Molden, Open Babel, \dots} (chart-2-1);
  \draw
  (chart-2-1) edge node [right] {\texttt{g16.prepare}} (chart-3-1);
  \draw
  (chart-3-1) edge node [right] {\texttt{g16.submit}} (chart-4-1);
  \draw
  (chart-4-2) -- +(0,-1) -| (chart-5-1) ;
  \draw
  (chart-5-1) -- +(-4,0) node [near start, above] {\textbf{no}} |- (chart-2-1) ;
  \draw
  (chart-5-1) edge node [right] {\texttt{g16.freqinput}} node [left] {\textbf{yes}} (chart-6-1);
  \draw
  (chart-4-1) edge node [above] {\texttt{g16}} (chart-4-2);
  \draw
  (chart-4-2) edge node [right] {\texttt{g16.chk2xyz}} (chart-5-2);
  \draw
  (chart-4-2) -| (chart-5-3) node [near end, right] {\texttt{g16.getenergy}} ;
  \draw
  (chart-6-1) edge node [right] {\texttt{g16.submit}} (chart-7-1);
  \draw
  (chart-7-1) edge node [right] {inspect \texttt{*.log}} (chart-9-1);
  \draw
  (chart-7-1) edge node [above] {\texttt{g16}} (chart-7-2);
  \draw
  (chart-7-2) edge node [right] {\texttt{g16.chk2xyz}} (chart-8-2);
  \draw
  (chart-7-2) -| (chart-8-3) node [near end, right] {\texttt{g16.getfreq}} ;
  \draw
  (chart-9-1) edge node [above] {\(0\)} (chart-9-2);
  \draw
  (chart-9-2) edge (chart-9-3);
  \draw
  (chart-9-1) |- (chart-10-2) node [near start, right] {\(1\)} ;
  \draw
  (chart-10-2) edge (chart-10-3);
  \draw
  (chart-9-1) -- +(-4,0) node [near start, above] {\(\geq2\)} |- (chart-2-1)
  node [near start, sloped, above] {extract molecular structure, repeat optimisation};
  \draw
  (chart-9-3) -- +(+4,0) node [near start, above] {\textbf{no}}
  |- (chart-2-1) node [near start, sloped, below] {extract molecular structure, repeat optimisation};
  \draw
  (chart-10-3) -- +(0,-2) node [near start, right] {\textbf{no}}
  -| ($(chart-2-1)+(-4,0)$) node [near start, sloped, above] {extract molecular structure, modify/scan, repeat optimisation} -- (chart-2-1);
  \draw
  (chart-9-3) -- +(0,-2) node [near start, right] {\textbf{yes}} -| (chart-11-4);
  \draw
  (chart-10-3) -| (chart-11-4) node [near start, sloped, above] {\textbf{yes}}; 
\end{tikzpicture}

\end{document}
