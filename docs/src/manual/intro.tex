\documentclass[   % 
  final,          % 
  a4paper         % 
]{article}

\usepackage[a4paper,margin=2cm]{geometry}
\setcounter{secnumdepth}{0}
\usepackage{hyperref}
\usepackage{listings}

\begin{document}
\section{Introduction}

The essence of quantum chemistry is performing calculations with a variety of different programs.
Quite a few of these use low level input and output systems.
In principle you are dealing with quite a few text-based files.

It is in the nature of the problem, 
that some of the regular tasks are quite repetitive.
In order to ease these repetitive tasks,
but more importantly also secure a certain level of consistency,
some steps can be automated.
This is where this repository, 
\href{https://github.com/polyluxus/tools-for-g16.bash}{tools-for-g16.bash},
comes into play.
Within it are contained a few scripts written in bash,
which are targeted to aid the use of the quantum chemistry software package 
\href{http://gaussian.com/}{Gaussian 16}.
They cover automatic generation of input files,
submissing of these to a queueing system,
and utilities to archive and extract results.

Please understand that this project started out to help me with my everyday work. 
I am happy to hear about suggestions and bugs. 
I am also fairly certain, that it is and will be a work in progress for quite some time.
Be aware that it could be in constant flux. 

This 'software' comes with absolutely no warrenty. None. Nada.
If you decide to use any of the scripts, it is entirely your resonsibility.

\subsection{Preliminary notes on the usage of this manual}

To make the manual easier to understand a few abreviations 
and common notations are used:
Anything set in brackets \lstinline{[ ]} indicates optional arguments or inputs.
Arguments in angles \lstinline{< >} require human input, 
and a vertical bar \lstinline{|} indicates alternatives between the inputs or options.

The following abbreviations will be used throughout:

\begin{tabular}{p{0.1\linewidth}p{0.8\linewidth}}
  {\lstinline!opt!} & Short for option(s) \\
  {\lstinline!ARG!} & String type argument \\
  {\lstinline!INT!} & Positive integer (including zero) \\
  {\lstinline!NUM!} & Signed whole number (including zero) \\
  {\lstinline!FLT!} & Floating point number \\
  {\lstinline!DUR!} & Duration in format {\lstinline![[HH:]MM:]SS!}  \\
\end{tabular}
\end{document}
