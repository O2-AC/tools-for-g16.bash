\documentclass[   % inherits article
  final,          % or
  % draft,
  a4paper,        % or
  % letterpaper   % (etc. pp.)
  rscols=3,       % this is the default
  margin=1.0cm,   % default, or
]{refsheet}

% Typesetting codeprettier
\usepackage[dvipsnames]{xcolor}
\definecolor{lightgray}{gray}{0.95}
\usepackage{listings}
\lstdefinestyle{mytex}{
  language=[LaTeX]TeX,%
  backgroundcolor=\color{lightgray},%
  basicstyle=\footnotesize\ttfamily,%
  keywordstyle=\bfseries,%
  breaklines=true,%
  morekeywords={part,chapter,subsection,subsubsection,paragraph,subparagraph}%
}

\lstset{style=mytex}

% Symbols for SE and GH
\usepackage{fontawesome} 
% Symbols for creative commons
\usepackage{ccicons}
% linking to the outside world
\usepackage{hyperref}
% Martin's weird SE handle
% (use a different package and Xe/LuaLaTeX if you're doing proper Japanese cheat sheets.)
\usepackage{CJKutf8} 

% Please no font errors
% https://tex.stackexchange.com/a/9366/33413
\newcommand\mybackslash{\char`\\}

% What this is about
\title{Cheat-Sheet for \texttt{tools-for-g16.bash} (0.0.17, 2018-10-09)}
\author{Martin C Schwarzer}
\date{\today}

\begin{document}
\maketitle

\section{Introduction}

This accompanies the repository \href{https://github.com/polyluxus/tools-for-g16.bash}{polyluxus/tools-for-g16.bash}.

Various bash scripts to aid the use of the quantum chemistry software package Gaussian 16.

\subsection{Preliminary notes}

The notation in brackets \texttt{[ ]} indicate optional arguments/inputs;
arguments in angles \texttt{< >} require human input;
a bar \texttt{|} indicates alternatives.

The following abbreviations will be used:
\begin{rslisttt}{ARG}
  \item[opt] Short for option(s)
  \item[ARG] String type argument
  \item[INT] Positive integer (including zero)
  \item[NUM] Whole number (including zero)
  \item[FLT] Floating point number
  \item[DUR] Duration in format \texttt{[[HH:]MM:]SS} 
\end{rslisttt}

\subsection{Installation \& Configuration}

General settings for the scripts can be found in the file \texttt{g16.tools.rc}.
Alternatively, settings can be stored in \texttt{.g16.toolsrc}, 
which always has precedence.
Every script will check three different directories in the order
\begin{enumerate*}
\item installation directory
\item user's home
\item parent working directory.
\end{enumerate*}
It will load the last configuration file it finds.

Setting files can be generated with the \texttt{configure/configure.sh} script.

\section{\texttt{g16.prepare.sh}}

This tool reads in a file containing a set of cartesian coordinates and
writes a Gaussian inputfile with predefined keywords. 
The script interfaces to Xmol format, 
Turbomole/ GFN-xTB \texttt{coord} format, too.

Usage: \texttt{g16.prepare.sh [opt] <file>}

\begin{rslisttt}{-m <ARG>}
  \item[-T <FLT>] Temperature (kelvin)
  \item[-P <FLT>] Pressure (atmosphere)
  \item[-r <ARG>] Add \texttt{ARG} to route section
  \item[-R <ARG>] Specific route section \texttt{ARG}
  \item[-l <INT>] Load predefined route section 
  \item[-l list ] Show all predefined route sections
  \item[-t <ARG>] Adds \texttt{ARG} to end of file
  \item[-C <ARG>] Specify caption/title of job;\\ Replacements:
  \begin{rsinline} 
    \item[\%F] input filename
    \item[\%f] input filename without \texttt{.xyz}
    \item[\%s] like \texttt{\%f}, also filtering \texttt{start}
    \item[\%j] jobname%
    \item[\%c] charge (with indicator \texttt{chrg})
    \item[\%M] multiplicity (with indicator \texttt{mult})
    \item[\%U] unpaired electrons (with indicator \texttt{uhf})%
  \end{rsinline}
  \item[-j <ARG>] Jobname 
  \item[-j \%s  ] Jobname is filename filtering \texttt{start.xyz}
  \item[-f <ARG>] Filename of generated input
  \item[-c <NUM>] Charge 
  \item[-M <INT>] Multiplicity (\( \geq 1 \))
  \item[-U <INT>] Unpaired electrons (\( \geq 0 \))
  \item[-m <INT>] Memory (megabyte)
  \item[-p <INT>] Processors
  \item[-d <INT>] disksize via \texttt{MaxDisk} (megabyte)
  \item[-s      ] Silence script (incremental)
  \item[-h      ] Help file 
\end{rslisttt}

\section{\texttt{g16.testroute.sh}}

This tool parses a Gaussian 16 inputfile and 
tests the route section for syntax errors 
with the Gaussian 16 utility \texttt{testrt}.

\begin{rslisttt}{-m <ARG>}
  \item[-s      ] Silence script (incremental)
  \item[-h      ] Help file 
\end{rslisttt}

\section{\texttt{g16.dissolve.sh}}

This tool reads in a Gaussian 16 inputfile and 
adds relevant keywords for solvent corrections.

Usage: \texttt{g16.dissolve.sh [opt] <file>}

\begin{rslisttt}{-m <ARG>}
  \item[-o <ARG>] Adds option \texttt{ARG} to the \texttt{scrf} keyword.
  \item[-S <ARG>] Adds option \texttt{solvent=ARG} to the \texttt{scrf} option list.
  \item[-O      ] Runs an optimisation (preserves or adds \texttt{OPT})
  \item[-r <ARG>] Add \texttt{ARG} to route section
  \item[-t <ARG>] Adds \texttt{ARG} to end of file
  \item[-m <INT>] Memory (megabyte)
  \item[-p <INT>] Processors
  \item[-d <INT>] disksize via \texttt{MaxDisk} (megabyte)
  \item[-s      ] Silence script (incremental)
  \item[-h      ] Help file 
\end{rslisttt}

\section{\texttt{g16.freqinput.sh}}

This tool reads in a Gaussian 16 inputfile and 
adds relevant keywords for a frequency calculation.

Usage: \texttt{g16.freqinput.sh [opt] <file>}

\begin{rslisttt}{-m <ARG>}
  \item[-o <ARG>] Adds option \texttt{ARG} to the \texttt{freq} keyword.
  \item[-R      ] Adds option \texttt{ReadFC} to the \texttt{freq} option list.
  \item[-T <FLT>] Temperature (kelvin)
  \item[-P <FLT>] Pressure (atmosphere)
  \item[-r <ARG>] Add \texttt{ARG} to route section
  \item[-t <ARG>] Adds \texttt{ARG} to end of file
  \item[-m <INT>] Memory (megabyte)
  \item[-p <INT>] Processors
  \item[-d <INT>] disksize via \texttt{MaxDisk} (megabyte)
  \item[-s      ] Silence script (incremental)
  \item[-h      ] Help file 
\end{rslisttt}

\section{\texttt{g16.ircinput.sh}}

This tool reads in a Gaussian 16 inputfile from a frequency run and 
adds relevant keywords for two separate irc calculations.

Usage: \texttt{g16.ircinput.sh [opt] <file>}

\begin{rslisttt}{-m <ARG>}
  \item[-o <ARG>] Adds option \texttt{ARG} to the \texttt{irc} keyword.
  \item[-r <ARG>] Add \texttt{ARG} to route section
  \item[-t <ARG>] Adds \texttt{ARG} to end of file
  \item[-m <INT>] Memory (megabyte)
  \item[-p <INT>] Processors
  \item[-d <INT>] disksize via \texttt{MaxDisk} (megabyte)
  \item[-s      ] Silence script (incremental)
  \item[-h      ] Help file 
\end{rslisttt}

\section{\texttt{g16.optinput.sh}}

This tool reads in a Gaussian 16 inputfile preferably from an IRC run and 
writes and inputfile for a subsequent structure optimisation.

Usage: \texttt{g16.optinput.sh [opt] <file>}

\begin{rslisttt}{-m <ARG>}
  \item[-o <ARG>] Adds option \texttt{ARG} to the \texttt{opt} keyword.
  \item[-r <ARG>] Add \texttt{ARG} to route section
  \item[-t <ARG>] Adds \texttt{ARG} to end of file
  \item[-m <INT>] Memory (megabyte)
  \item[-p <INT>] Processors
  \item[-d <INT>] disksize via \texttt{MaxDisk} (megabyte)
  \item[-s      ] Silence script (incremental)
  \item[-h      ] Help file 
\end{rslisttt}

\newpage

\section{\texttt{g16.submit.sh}}

This tool parses and then submits a Gaussian 16 inputfile to a queueing system.

Usage: \texttt{g16.submit.sh [opt] <file>}

\begin{rslisttt}{-m <ARG>}
  \item[-m <INT>] Memory (megabyte)
  \item[-p <INT>] Processors
  \item[-d <INT>] disksize via \texttt{MaxDisk} (megabyte)
  \item[-w <DUR>] Walltime limit
    %\item[-b <ARG>] Specify binary (TODO)
  \item[-e <ARG>] Specify an environment variable \texttt{ARG} in format \texttt{<VAR=value>}
  \item[-j <INT>] Wait for job with ID \texttt{INT}
  \item[-H      ] Submit with status hold (PBS) or \texttt{PSUSP} (BSUB)
  \item[-k      ] Only create (keep) the jobscript, do not submit it.
  \item[-Q <ARG>] Queue for which job script should be created 
    (\texttt{pbs-gen}/\texttt{bsub-rwth})
    %\item[-q <ARG>] Specify queue (TODO)
  \item[-P <ARG>] Account to project (BSUB);
    if \texttt{ARG} is \texttt{default}/\texttt{0}/\texttt{''} presets are overwritten.
  \item[-u <ARG>] set user email address (BSUB);
    if \texttt{ARG} is \texttt{default}/\texttt{0}/\texttt{''} presets are overwritten.
  \item[-s      ] Silence script (incremental)
  \item[-h      ] Help file 
\end{rslisttt}

\section{\texttt{g16.getenergy.sh}}

This tool finds energy statements from Gaussian 16 calculations.

Usage: \texttt{g16.getenergy.sh [opt] [<file(s)>]}

If no files given, it finds energy statements from all log files in the current directory.

\begin{rslisttt}{-m <ARG>}
  \item[-i <ARG>] Specify input suffix if processing directory
  \item[-o <ARG>] Specify output suffix if processing directory
  \item[-s      ] Silence script (incremental)
  \item[-h      ] Help file 
\end{rslisttt}

\section{\texttt{g16.getfreq.sh}}

This tool summarises a frequency calculation and extracts the thermochemistry data.

Usage: \texttt{g16.getfreq.sh [opt] <file(s)>}

\begin{rslisttt}{-m <ARG>}
  \item[-v      ] Incrementally increase verbosity
  \item[-V <INT>] Set level of verbosity directly, (0-4)
  \item[-c      ] Separate values by comma (\texttt{-V0} or \texttt{-V1})
  \item[-O <ARG>] Write summary to file instead of screen
  \item[-s      ] Silence script (incremental)
  \item[-h      ] Help file 
\end{rslisttt}

\section{\texttt{g16.chk2xyz.sh}}

A tool to convert a checkpoint file to an \texttt{xyz} file. 
This formats the \texttt{chk} first to a \texttt{fchk}.

Usage: \texttt{g16.chk2xyz.sh [-s] -h | -f | <chk-file(s)>}

\begin{rslisttt}{-m <ARG>}
  \item[-f      ] Formats all checkpointfiles that are found in the current directory
  \item[-s      ] Silence script (incremental)
  \item[-h      ] Help file 
\end{rslisttt}

% move to end
\vfill
\section{Author, Bugs, and the Rest}

\begin{flushright}
Martin C Schwarzer  
\faStackExchange~\href{https://chemistry.stackexchange.com/users/4945}{Martin-\begin{CJK*}{UTF8}{min}マーチン\end{CJK*}}
\faGithub~\href{https://github.com/polyluxus}{polyluxus}

This document is licensed \ccbysa.
\end{flushright}

%%because there currently is not enough to fill the second page
%\columnbreak
%\null\vfill\null
%\columnbreak
%\null\vfill\null

\end{document}
